\documentclass{article}
 
\usepackage[T2A]{fontenc}
\usepackage[utf8]{inputenc}
\usepackage[russian]{babel}
\usepackage{hyphenat}
\hyphenation{ма-те-ма-ти-ка вос-ста-нав-ли-вать}

\usepackage[T1]{fontenc}
\usepackage{babel}

\author{
  Колодезева Екатерина\\
  \texttt{kolodezeva.ed@phystech.edu}
}
\title{Теория поля, Воронов, 2018/19}

\begin{document}

\maketitle

\begin{abstract}
  Данный документ представляет собой конспект лекций по семестровому курсу теории поля, прочитанному на лекциях в МФТИ Вороновым Николаем Александровичем осенью 2018/19 учебного года студентам третьего курса ФАКИ.
\end{abstract}
 
\tableofcontents
 
\section{Предисловие}
 Этот текст будет на русском языке. Это демонстрация того, что символы кириллицы
 в сгенерированном документе (Compile to PDF) отображаются правильно.
 Для этого Вы должны установить нужный  язык (russian) 
и необходимую кодировку шрифта (T2A).
 
\section{Математические формулы}
Кириллические символы также могут быть использованы в математическом режиме.
 
\begin{equation}
  S_\textup{ис} = S_{123}
\end{equation}

\author{Колодезева Екатерина, гр. 635}
 
\end{document}
