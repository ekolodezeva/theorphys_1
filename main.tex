\documentclass{article}
 
\usepackage[T2A]{fontenc}
\usepackage[utf8]{inputenc}
\usepackage[russian]{babel}
\usepackage{hyphenat}
\hyphenation{ма-те-ма-ти-ка вос-ста-нав-ли-вать}

\usepackage{amsmath}
\usepackage{mathtools}
\usepackage{tensor}
\DeclarePairedDelimiter\abs{\lvert}{\rvert}%

\usepackage{hyperref}

\author{
  Колодезева Екатерина\\
  \texttt{kolodezeva.ed@phystech.edu}
}
\title{Теория поля (Воронов) 2018/19}

\begin{document}

\maketitle

\begin{abstract}
  Данный документ представляет собой конспект лекций по семестровому курсу теории поля в МФТИ, прочитанному Вороновым Николаем Александровичем осенью 2018/19 учебного года студентам третьего курса ФАКИ. На достоверность изложения и отсутствие ошибок не претендует. Последняя версия документа доступна по ссылке \url{https://github.com/ekolodezeva/theorphys_1}.
\end{abstract}
 
\tableofcontents
 
\section{Вступление}
 
\subsection{Теория Ньютона}

В инерциальных системах отсчёта выполняются законы Ньютона. В предположениях законов Ньютона взаимодействие распространяется мгновенно, что видно из уравнений кулоновского и гравитационного притяжения:

\begin{equation*}
    m_a \frac{d^2\vec r_a}{dt^2}= \sum_{b}\frac{q_a q_b}{\abs{\vec r_a - \vec r_b}^3}(\vec r_a - \vec r_b)
\end{equation*}

\begin{equation*}
    m_a \frac{d^2\vec r_a}{dt^2}= G \sum_{b}\frac{m_a m_b}{\abs{\vec r_a - \vec r_b}^3}(\vec r_b - \vec r_a)
\end{equation*}

Преобразование координат при переходе между системами отсчёта в механике Ньютона в векторной и тензорной записях:

\begin{equation}\label{eq:newton}
    \begin{split}
        \vec r' = R\vec r+\vec vt+\vec r_0,\quad t'=t+t_0\\
        r_a' = \sum_{b} R_{ab}r_a+v_at+r_{0a},\quad t'=t+t_0,\\
        \textup{R — ортогональная матрица}
    \end{split}
\end{equation}

Рассмотрим некоторые частные случаи \eqref{eq:newton}:
\begin{enumerate}
    \item Однородность пространства и времени в механике Ньютона: пусть $R=I, \vec{v}=\vec{0}$. Тогда $\vec{r}'=\vec{r}+\vec{r_0}, t'=t+t_0$.
    \item Изотропность пространства (поворот): пусть $R\neq{I}, \vec{v}=\vec{0}, t_0=0$. Тогда $\vec{r}'=R\vec{r}$, $t'=t$.
    \item Преобразование Галилея: пусть $R=I, \vec{r_0}=\vec{0}, t_0=0$. Тогда $\vec{r}'=\vec{r}+\vec{v}t, t'=t$.
\end{enumerate}

Принцип относительности систем отсчёта в классической механике гласит, что все системы отсчёта равноправны.

%chapter
\section{Кинематика в специальной теории относительности}

\subsection{Принцип относительности Эйнштейна}

Принцип относительности Эйнштейна гласит, что и в электродинамике законы природы одни и те же для всех систем отсчёта. На замену преобразованию Галилея приходит преобразование Лоренца.

Знакомые уравнения электродинамики будут заменой переменных переведены к виду, иллюстрирующему этот принцип. Рассмотрим систему уравнений Максвелла:

\begin{equation}\label{eq:maxwell}
    \begin{cases}
    div\vec{E}=4\pi\rho\\
    div\vec{H}=0\\
    rot\vec{E}=-\frac{1}{c}\frac{\partial\vec{H}}{\partial t}\\
    rot\vec{H}=\frac{1}{c}\frac{\partial\vec{E}}{\partial t}+\frac{4\pi}{c}\vec{j}
    \end{cases}
\end{equation}

Система переопределена: она содержит 8 уравнений и 6 неизвестных ($\vec{E}$ и $\vec{H}$
), её уравнения неинвариантны относительно преобразования Лоренца. В последующих лекциях будет проведена замена переменных

\begin{equation}
    \begin{cases}
    \vec{H}=rot\vec{A}\\
    \vec{E}=-\frac{1}{c}\frac{\partial\vec{A}}{\partial t}-\nabla\varphi
    \end{cases}
\end{equation}

С помощью этой замены уравнения \eqref{eq:maxwell} перейдут в систему из 4 уравнений с 4 неизвестными $\varphi,\vec{A}$:

\begin{equation}\label{eq:maxwell_a}
    \begin{cases}
    \frac{1}{c^2}\frac{\partial^2}{\partial t^2}\vec{A}-\Delta\vec{A}=\frac{4\pi}{v}\vec{j}\\
    \frac{1}{c^2}\frac{\partial^2}{\partial t^2}\varphi-\Delta\varphi=4\pi\rho
    \end{cases}
\end{equation}

Уравнения \eqref{eq:maxwell_a} инвариантны относительно преобразований Лоренца.

\subsection{Преобразование координат}

Рассмотрим две системы отсчёта. Система $K'$ для наблюдателя покоится, система $K$ движется относительно её со скоростью $v$, направленной вдоль оси $x$. Тогда координаты точки в системе отсчёта $K'$ зависят от её координат в системе отсчёта $K$ следующим образом:

\begin{equation}\label{eq:lorentz_coord}
    \begin{cases}
    x'=\frac{x-vt}{\sqrt{1-\frac{v^2}{c^2}}}=\gamma\left(x-vt\right)\\
    t'=\frac{t-vx/c^2}{\sqrt{1-\frac{v^2}{c^2}}}=\gamma\left(t-vx/c^2\right)\\
    y'=y\\
    z'=z
    \end{cases}
\end{equation}


Преобразование Лоренца — это поворот в псевдоевклидовом четырёхмерном пространстве Минковского. В пространстве Минковского расстояние между двумя точками $\left(\vec{r_1}, t_1\right)$ и $\left(\vec{r_2}, t_2\right)$ называется интервалом и задаётся формулой
\begin{equation}
    S_12=c^2\left(t_1-t_2\right)^2-\left(\vec{r_1}-\vec{r_2}\right)^2
\end{equation}

Матрицу преобразования только одной координаты и времени легко получить как матрицу поворота, сохраняющую метрику в пространстве Минковского из следующей системы уравнений:

\begin{equation}\label{eq:lorentz_x}
    \begin{cases}
    dx'=Adx+Bdt\\
    dt'=Ddx+Fdt\\
    c^2dt'^2-dx'^2=c^2dt^2-dx^2
    \end{cases}
\end{equation}

Подставив первые два уравнения в третье, получаем

\begin{equation*}
    c^2\left(Dx+Fdt\right)^2-\left(Adx+Bdt\right)^2=c^2dt^2-dx^2
\end{equation*}
\begin{equation*}
    c^2D^2dx^2+c^2F^2dt^2+2c^2DFdxdt-A^2dx^2-B^2dt^2-2ABdxdt=c^2dt^2-dx^2
\end{equation*}
\begin{equation*}
    dx^2\left(c^2D^2-A^2+1\right)^2+dt^2\left(c^2F^2-B^2-c^2\right)^2+2dxdt\left(DF-AB\right)=0
\end{equation*}

\begin{equation}\label{eq:lorentz_param}
    \begin{cases}
    c^2D^2=A^2-1\\
    B^2=c^2\left(F^2-1\right)\\
    c^2DF=AB
    \end{cases}
\end{equation}

Система \eqref{eq:lorentz_param} содержит 3 уравнения и четыре неизвестных параметра $A, B, D, F$. Добавим необходимое уравнение, введя известный параметр $v$:

\begin{equation*}
    B=FV
\end{equation*}

Решая систему, получаем

\begin{equation*}
    F^2V^2=c^2\left(F^2-1\right) \implies F=\frac{1}{\sqrt{V^2-c^2}}
\end{equation*}
\begin{equation*}
    B=FV=\frac{V}{\sqrt{V^2-c^2}}
\end{equation*}
\begin{equation*}
    \begin{cases}
    V=\frac{B}{F}=\frac{c^2D}{A} \implies D=\frac{AV}{c^2}\\
    c^2D^2=A^2-1 \implies \frac{A^2V^2}{c^2}=A^2-1 \implies A=\frac{1}{\sqrt{V^2-c^2}}
    \end{cases}
\end{equation*}
\begin{equation*}
    D=\frac{AV}{c^2}=\frac{V/c^2}{\sqrt{V^2-c^2}}
\end{equation*}

Итоговый вид преобразования \eqref{eq:lorentz_x}:

\begin{equation*}
    \begin{cases}
    dx'=\frac{1}{\sqrt{V^2-c^2}}\big(dx+Vdt\big)\\
    dt'=\frac{1}{\sqrt{V^2-c^2}}\big(\frac{V}{c^2}dx+dt\big)
    \end{cases}
\end{equation*}

Видно, что физический смысл переменной $V$ — это введённая выше относительная скорость $v$ систем отсчёта $K$ и $K'$.

...

\subsection{Преобразование скорости}

...

\subsection{Элементы векторного анализа}

...

\subsection{Лагранжиан и действие для свободной точечной релятивистской частицы}

...

%chapter
\section{Заряд в электромагнитном поле}

\subsection{Векторный потенциал электромагнитного поля}

...

\begin{equation}\label{eq:fmatrix}
    \tensor{F}{^i^k} =
    \begin{pmatrix}
        0 & -E_x & -E_y & -E_z \\
        E_x & 0 & -H_z & H_y \\
        E_y & H_z & 0 & -H_x \\
        E_z & -H_y & H_x & 0
    \end{pmatrix}
\end{equation}

\subsection{Взаимодействие заряженных релятивистских частиц в электромагнитном поле}

...

Вариация действия на минимальном пути равна нулю, значит, равно нулю и подынтегральное выражение. Получаем уравнение движения частицы во внешнем электромагнитном поле:

\begin{equation}\label{eq:movement}
    mc\frac{d\tensor{u}{^i}}{ds}=\frac{e}{c}\tensor{F}{^i^k}\tensor{u}{_k}
\end{equation}

Полученное тензорное уравнение содержит в себе четыре скалярных уравнения и четыре неизвестных $\tensor{u}{^i}$. Ранее мы выяснили, что 

\begin{equation}\label{eq:velnorm}
    \tensor{\eta}{_i_k}\tensor{u}{^i}\tensor{u}{^k}=1
\end{equation}

Убедимся, что это уравнение не противоречит системе уравнений \eqref{eq:movement} и не переопределяет её. Продифференцируем его по $ds$:

\begin{equation*}
    \begin{split}
        \frac{d}{ds}\big(\tensor{\eta}{_i_k}\tensor{u}{^i}\tensor{u}{^k}\big)=\tensor{\eta}{_i_k}\frac{d\tensor{u}{^i}}{ds}\tensor{u}{^k}+\tensor{\eta}{_i_k}\tensor{u}{^i}\frac{d\tensor{u}{^k}}{ds}=\tensor{\eta}{_i_k}\frac{d\tensor{u}{^i}}{ds}\tensor{u}{^k}+\tensor{\eta}{_k_i}\tensor{u}{^k}\frac{d\tensor{u}{^i}}{ds}=\\
        =2\tensor{\eta}{_i_k}\frac{d\tensor{u}{^i}}{ds}\tensor{u}{^k}=2\frac{d\tensor{u}{^i}}{ds}\tensor{u}{_i}=0
    \end{split}
\end{equation*}

Подставляя выражение для $d\tensor{u}{^i}/ds$ из \eqref{eq:movement} с учётом равенства нулю свёртки антисимметричного тензора $\tensor{F}{^i^k}$ и симметричного тензора $\tensor{u}{_k}\tensor{u}{_i}$, получаем

\begin{equation*}
    \frac{d\tensor{u}{^i}}{ds}\tensor{u}{_i}=\frac{e}{mc^2}\tensor{F}{^i^k}\tensor{u}{_k}\tensor{u}{_i}=0 
\end{equation*}

Таким образом, условие \eqref{eq:velnorm} не противоречит и не переопределяет систему уравнений \eqref{eq:movement}.

Получим из \eqref{eq:movement} уравнения движения частицы в трёхмерных координатах, подставив вместо $\tensor{F}{^i^k}$ компоненты полей $\vec{E}$ и $\vec{H}$ в соответствии с \eqref{eq:fmatrix}:

\begin{equation*}
    mc\frac{d\tensor{u}{^0}}{ds}=\frac{d\left(mc\tensor{u}{^0}\right)}{ds}=\frac{d\left(\varepsilon/c\right)}{ds}=\frac{1}{c}\frac{d\varepsilon}{dt}\frac{dt}{ds}=\Biggm\lvert\frac{ds}{dt}=\frac{\sqrt{c^2dt^2-d\vec{r}^2}}{dt}=\sqrt{c^2-v^2}=\frac{c}{\gamma}\Biggm\rvert=\frac{\gamma}{c^2}\frac{d\varepsilon}{dt}=
\end{equation*}
\begin{equation*}
    =\frac{e}{c}\tensor{F}{^0^k}\tensor{u}{_k}=\frac{e}{c}\big(\tensor{F}{^0^0}\tensor{u}{_0}-\tensor{F}{^0^\alpha}\tensor{u}{_\alpha}\big)    =-\frac{e}{c}\left(-E_\alpha\right)\frac{v_\alpha\gamma}{c}=\frac{\gamma e}{c^2}\left(\vec{E}\vec{v}\right)
\end{equation*}

\begin{equation}\label{eq:emovement}
    \frac{d\varepsilon}{dt}=e\left(\vec{E}\vec{v}\right)
\end{equation}

\begin{equation*}
    mc\frac{d\tensor{u}{^\alpha}}{ds}=\frac{d\left(mc\tensor{u}{^\alpha}\right)}{ds}=\frac{dp}{ds}=\frac{\gamma}{c}\frac{dp}{dt}=\frac{e}{c}\tensor{F}{^\alpha^k}\tensor{u}{_k}=\frac{e}{c}\big(\tensor{F}{^\alpha^0}\tensor{u}{_0}-\tensor{F}{^\alpha^\beta}\tensor{u}{_\beta}\big)=
\end{equation*}
\begin{equation*}
    =\frac{e}{c}\big(E_\alpha\gamma-\tensor{e}{_\gamma_\alpha_\beta}\tensor{H}{_\alpha}\tensor{v}{_\beta}\frac{\gamma}{c}\big)=\frac{\gamma}{c}\big(eE_\alpha+\frac{e}{c}\left(\lbrack\vec{v}\times\vec{H}\rbrack\right)_\alpha\big)
\end{equation*}

\begin{equation}\label{eq:hmovement}
    \frac{dp}{dt}=e\vec{E}+\frac{e}{c}\lbrack\vec{v}\times\vec{H}\rbrack
\end{equation}

Система уравнений \eqref{eq:emovement}, \eqref{eq:hmovement} эквивалентна уравнению \eqref{eq:movement}.

\subsection{Преобразование Лоренца для полей}

Тензор электромагнитного поля

\begin{equation}
    \tensor{F}{^i^k} =
    \begin{pmatrix}
        0 & -E_x & -E_y & -E_z \\
        E_x & 0 & -H_z & H_y \\
        E_y & H_z & 0 & -H_x \\
        E_z & -H_y & H_x & 0
    \end{pmatrix}
\end{equation}

Ковариантный вектор преобразуется по правилу $x^{i'}=\tensor{\Lambda}{^{i'}_{i}} x^i$. Значит, дважды ковариантный тензор $\tensor{F}{^i^k}$ преобразуется как $\tensor{F}{^{i'}^{k'}}=\tensor{\Lambda}{^{i'}_{k'}}\tensor{\Lambda}{^{k'}_{k}} \tensor{F}{^i^k}$.

...

\subsection{Теорема Пойнтинга — Умова}

...

%chapter
\section{Постоянное электромагнитное поле}

\subsection{Электрическое поле}

...

\end{document}
